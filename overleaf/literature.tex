% ---- ETD Document Class and Useful Packages ---- %
\documentclass{ucetd}
\usepackage{subfigure,epsfig,amsfonts}
\usepackage{natbib}
\usepackage{amsmath}
\usepackage{amssymb}
\usepackage{amsthm}
\usepackage{caption}
\usepackage{titlesec}
\usepackage[nottoc]{tocbibind} %Includes "References" in the table of contents
\usepackage{graphicx} %package to manage images

% set figures path
\graphicspath{ {./figures/} }
% left justify sections and subsections
\titleformat{\section} 
	{\normalfont\Large\bfseries}{\makebox[30pt][l]{\thesection}}{0pt}{} 
\titleformat{\subsection} 
	{\normalfont\large\bfseries}{\makebox[30pt][l]{\thesubsection}}{0pt}{}

%% Use these commands to set biographic information for the title page:
\title{Thesis Title}
\author{Thesis Author}
\department{Thesis Department}
\division{Thesis Division}
\degree{Type of Degree}
\date{Graduation Date}

%% Use these commands to set a dedication and epigraph text
\dedication{Dedication Text}
\epigraph{Epigraph Text}


\begin{document}
%% Basic setup commands
% If you don't want a title page comment out the next line and uncomment the line after it:
\maketitle
%\omittitle

% These lines can be commented out to disable the copyright/dedication/epigraph pages
\makecopyright
\makededication
\makeepigraph

%% Make the various tables of contents
\tableofcontents
\listoffigures
\listoftables

\acknowledgments
% Enter Acknowledgements here

\abstract
% Enter Abstract here

\mainmatter
% Main body of text follows

\chapter{Medical Expenditure Panel Survey}
\section{Introduction}
\section{Open Source}
% Introductory stuff

\chapter{Population Overview}
\section{Expenditures} \label{sec_expenditures}
The United States spends a larger share of its Gross Domestic Product on health care than any other major industrialized country and performed less well on many population outcomes.\cite{papanicolas} To address high health care costs, policymakers need to understand how costs are distributed across different types of services, different payers and by selected characteristics of the population. In this section we present estimates of the percentage of persons with health care expenses, mean and median expenses by type of service, age and insurance status for the U.S. civilian non institutionalized population over the past 10 years (2008-2018). All expenditure metrics in this section have been adjusted to reflect their equivalent 2018 value. All differences between estimates discussed in this section are statistically significant at the 0.05 level.

Following the methodology outlined in the AHRQ Statistical Brief \#493, we have group services into the following categories. Hospital inpatient services, which is generally directed toward serious ailments and trauma requiring one of more days of overnight stay at a hospital. Ambulatory services, which refers to medical services performed on an outpatient basis, without admission to a hospital or other facility. Office-based visits, hospital outpatient care, and emergency room services fall into the ambulatory services group. Home health care, other medical services and equipment, which include agency and non agency home health care events, all medical equipment including glasses and contacts and miscellaneous medical services such as ambulance services. In addition to these groups we also have two self explanatory groups prescribed medicines and dental services. \cite{statisticalbrief493}

Expenditures (expenses) include total direct payments from all sources to hospitals, physicians, home health providers (agency and paid independent providers), dental providers, other types of health care providers (e.g., physical therapists, chiropractors, optometrists, etc.) and pharmacies for services reported by respondents in the MEPS-HC. Expenditures for hospital-based services include those for both facility and separately billed physician services.

\subsection{ Persons with Health Care Expenses}
For each year from 2008 through 2018 we cycled through all the service categories described in the \ref{sec_expenditures}. Using the respondents weight variable and the respondents expenditure variables we were able to generate estimates for the percentage of the population with who had some expenses for health care services. The proportion of people with expenses varied widely by type of service, with large proportions having expenses for ambulatory services 78.1\% in 2018 (Figure~\ref{fig:pct_expenses_ambulatory}) and prescribed medicines 62.6\% in 2008 (Figure~\ref{fig:pct_expenses_prescribed_medicines}) and much smaller proportions having expenses for home health care and other medical services 17.9\% in 2012 (Figure~\ref{fig:pct_expenses_home_health_care_and_other_medical_services_and_equipment}), and hospital inpatient services 6.3\% in 2015 (Figure~\ref{fig:pct_expenses_hospital_inpatient}). 

In particular, 2018 was a year where multiple age groups and service categories had higher than usual proportions of people with expenses. The all ages and under 65 age groups had the highest values for any services, ambulatory services and dental services. Both the under 65 and over 65 age groups had the highest values for Home Health Care and Other Medical Services and Equipment in this year as well. All other variations in previous years, age groups and service types was statistically insignificant.

\subsection{ Mean and Median Health Care Expenses}
For each year from 2008 through 2018 we cycled through all the service categories described in the \ref{sec_expenditures}. Using the respondents weight variable and the respondents expenditure variables we were able to generate estimates for the mean and median annual expenditures for the population with who had some expenses with a particular health care service. 

The mean total expense per person for those with some health care expenses over the decade of analysis has remained relatively stable averaging at \$5,903. The mean total expenses for persons age 65 and older with some expenses (\$11,403) was more than twice the amount for persons under age 65 (\$4,833). Focusing on annual expenses per person we found that in 2018 the under 65 group had a significantly higher mean value for any services (\$5,741). (Figure~\ref{fig:mean_expenses_any_services}). Additionally in 2018 the all ages and over 65 age groups had a significantly higher mean value for dental services (\$1,135 and \$873 respectively) (Figure~\ref{fig:mean_expenses_dental_services}).  

Among specific health care service categories, the mean expense per person who had expenses of that service type ranged from \$19,806 for hospital inpatient services to \$774 for dental services . Adults age 65 and older had larger mean expenses in all service categories compared to persons under age 65 excluding hospital inpatient services for which the difference in means between these age groups was not significant (Figures ~\ref{fig:mean_expenses_any_services} ~\ref{fig:mean_expenses_hospital_inpatient} ~\ref{fig:mean_expenses_ambulatory} ~\ref{fig:mean_expenses_prescribed_medicines} ~\ref{fig:mean_expenses_dental_services} ~\ref{fig:mean_expenses_home_health_care_and_other_medical_services_and_equipment}). 

The overall median total expense per person with expenses was \$1,517, and ranged across service categories from \$10,993 for hospital inpatient services to less than \$400 for prescribed medicines, dental services, and home health care and other medical services and supplies (\$327, \$281, and \$310, respectively). Median expenses per person with any expense of that type were consistently higher for the elderly across all service categories when compared with persons under age 65, except for hospital inpatient expenses. (Figures ~\ref{fig:median_expenses_any_services} ~\ref{fig:median_expenses_hospital_inpatient} ~\ref{fig:median_expenses_ambulatory} ~\ref{fig:median_expenses_prescribed_medicines} ~\ref{fig:median_expenses_dental_services} ~\ref{fig:median_expenses_home_health_care_and_other_medical_services_and_equipment}). 

Focusing on annual expenses per person we found that in 2018 the under 65 and all ages groups had a significantly higher median value for any services (\$1,410 and \$1,850 respectively) (Figure ~\ref{fig:median_expenses_any_services}). Additionally those same groups had a significantly higher median value for home health care and other medical services and equipment (\$338 and \$379 respectively) (Figure ~\ref{fig:median_expenses_home_health_care_and_other_medical_services_and_equipment}).

\subsection{ Health Care Expenses by Insurance Coverage}
There are many degrees of resolution available for classifying insurance coverage using the MEPS-HC survey. Due to the scope of this research area we determined that the following categories were appropriate to capture broad patterns.

Individuals under age 65 were classified in the following three insurance categories:

\begin{itemize}
  \item \textit{private health insurance}: Individuals who, at any time during a given year, had insurance that provides coverage for hospital and physician care (other than Medicare, Medicaid, or other public coverage) were classified as having private insurance.
  \item \textit{public health insurance}: Individuals who, at all times during a given year, were not covered by private insurance and at any time during a given year were covered by any of the following public programs at any point during the year: Medicare, Medicaid, or other public coverage
  \item \textit{uninsured}: Individuals who, at all times during a given year, were not covered by private insurance or public insurance
\end{itemize}

Individuals age 65 and older were classified into the following three insurance categories:
\begin{itemize}
  \item \textit{medicare only}: This category includes individuals classified as Medicare beneficiaries but not classified as Medicare and private insurance or as Medicare and other public insurance.
  \item \textit{medicare and private insurance}: This category includes individuals classified as Medicare beneficiaries and covered by Medicare and a supplementary private policy.
  \item \textit{medicare and public insurance}: This category includes individuals classified as Medicare beneficiaries who were not covered by private insurance at any point during the year and were covered by a public programs at any point during the year including Medicaid.
\end{itemize}

Average annual health care expenses varied substantially by age and type of health insurance coverage. Persons under age 65 with expenses had a mean total expense of \$4,833 and a median total expense of \$1,174, while those 65 years and older had a mean total expense of \$11,403 and a median total expense of \$5,145. (Figures ~\ref{fig:insurance_expenses_mean_under_65} ~\ref{fig:insurance_expenses_median_under_65} ~\ref{fig:insurance_expenses_mean_over_65} ~\ref{fig:insurance_expenses_median_over_65}). In general the mean expenses for individuals with public or partial public insurance was higher than other individuals for both age groups. 

Among persons under age 65 with expenses, mean and median expenses had a significant peak in 2018 for those with private insurance. Among persons over age 65 with expenses, median expenses had a significant peak in 2018 for those with medicare and private insurance. 

\subsection{ Discussion}
One overarching theme of these analyses is the disparity. Expenditure disparity can be found between age groups. We have demonstrated that in multiple service types the over 65 age group has a higher percentage of persons with expenditures and a higher mean/median value per person. This is most clearly reflected when analyzing prescription medicines. A natural consequence of advances in health care increase the ability to manage multiple simultaneous chronic conditions, with consequent exponential growth in costs, while extending life expectancy. This type of relationship between underlying health, age and services may contribute to the differences identified.

In terms of expenditures this can be most easily noted by the differences between mean and median values. In general the mean values are much higher than median values because a relatively small portion persons account for a large proportion of expenses. The \textit{typical patient} is not representative of the average expenditures. Empirical distributions of healthcare expenditures is Pareto-like perhaps even worse than Pareto. For a Pareto-like distribution with \(\sigma \leq 2\) at large expenditures, the variance is not defined, and sample variance approaches infinity with increasing sample size. Therefore, unlike the case of distributions with finite variance, variability in the mean of a sample of size N does not decrease with N. This yields unique challenges with payment models as the traditional pooling of low risk members with high risk members becomes invalid. In further chapters we will demonstrate this and track the distribution across time. 


\section{Utilization}
% Intro to chapter one

\medskip
%Sets the bibliography style to UNSRT and imports the 
%bibliography file "bibliography.bib".
\bibliographystyle{unsrt}
\bibliography{bibliography}

\newpage
% Figures and tables, if you decide to leave them to the end

\begin{figure}[!ht]
\centering
\includegraphics[width=1\textwidth]{figures/pct_expenses/any_services}
\caption{Percentages of Persons with Any Services Expenses}
\label{fig:pct_expenses_any_services}
\end{figure}

\begin{figure}[!ht]
\centering
\includegraphics[width=1\textwidth]{figures/pct_expenses/hospital_inpatient}
\caption{Percentages of Persons with Hospital Inpatient Expenses}
\label{fig:pct_expenses_hospital_inpatient}
\end{figure}

\begin{figure}[!ht]
\centering
\includegraphics[width=1\textwidth]{figures/pct_expenses/ambulatory}
\caption{Percentages of Persons with Ambulatory Expenses}
\label{fig:pct_expenses_ambulatory}
\end{figure}

\begin{figure}[!ht]
\centering
\includegraphics[width=1\textwidth]{figures/pct_expenses/prescribed_medicines}
\caption{Percentages of Persons with Prescribed Medicines Expenses}
\label{fig:pct_expenses_prescribed_medicines}
\end{figure}

\begin{figure}[!ht]
\centering
\includegraphics[width=1\textwidth]{figures/pct_expenses/dental_services}
\caption{Percentages of Persons with Dental Services Expenses}
\label{fig:pct_expenses_dental_services}
\end{figure}

\begin{figure}[!ht]
\centering
\includegraphics[width=1\textwidth]{figures/pct_expenses/home_health_care_and_other_medical_services_and_equipment}
\caption{Percentages of Persons with Home Health Care and Other Medical Services and Equipment Expenses}
\label{fig:pct_expenses_home_health_care_and_other_medical_services_and_equipment}
\end{figure}

\begin{figure}[!ht]
\centering
\includegraphics[width=1\textwidth]{figures/mean_expenses/any_services}
\caption{Mean Expense Per Persons with Any Services Expenses}
\label{fig:mean_expenses_any_services}
\end{figure}

\begin{figure}[!ht]
\centering
\includegraphics[width=1\textwidth]{figures/mean_expenses/hospital_inpatient}
\caption{Mean Expense Per Persons with Hospital Inpatient Expenses}
\label{fig:mean_expenses_hospital_inpatient}
\end{figure}

\begin{figure}[!ht]
\centering
\includegraphics[width=1\textwidth]{figures/mean_expenses/ambulatory}
\caption{Mean Expense Per Persons with Ambulatory Expenses}
\label{fig:mean_expenses_ambulatory}
\end{figure}

\begin{figure}[!ht]
\centering
\includegraphics[width=1\textwidth]{figures/mean_expenses/prescribed_medicines}
\caption{Mean Expense Per Persons with Prescribed Medicines Expenses}
\label{fig:mean_expenses_prescribed_medicines}
\end{figure}

\begin{figure}[!ht]
\centering
\includegraphics[width=1\textwidth]{figures/mean_expenses/dental_services}
\caption{Mean Expense Per Persons with Dental Services Expenses}
\label{fig:mean_expenses_dental_services}
\end{figure}

\begin{figure}[!ht]
\centering
\includegraphics[width=1\textwidth]{figures/mean_expenses/home_health_care_and_other_medical_services_and_equipment}
\caption{Mean Expense Per Persons with Home Health Care and Other Medical Services and Equipment Expenses}
\label{fig:mean_expenses_home_health_care_and_other_medical_services_and_equipment}
\end{figure}

\begin{figure}[!ht]
\centering
\includegraphics[width=1\textwidth]{figures/median_expenses/any_services}
\caption{Median Expense Per Persons with Any Services Expenses}
\label{fig:median_expenses_any_services}
\end{figure}

\begin{figure}[!ht]
\centering
\includegraphics[width=1\textwidth]{figures/median_expenses/hospital_inpatient}
\caption{Median Expense Per Persons with Hospital Inpatient Expenses}
\label{fig:median_expenses_hospital_inpatient}
\end{figure}

\begin{figure}[!ht]
\centering
\includegraphics[width=1\textwidth]{figures/median_expenses/ambulatory}
\caption{Median Expense Per Persons with Ambulatory Expenses}
\label{fig:median_expenses_ambulatory}
\end{figure}

\begin{figure}[!ht]
\centering
\includegraphics[width=1\textwidth]{figures/median_expenses/prescribed_medicines}
\caption{Median Expense Per Persons with Prescribed Medicines Expenses}
\label{fig:median_expenses_prescribed_medicines}
\end{figure}

\begin{figure}[!ht]
\centering
\includegraphics[width=1\textwidth]{figures/median_expenses/dental_services}
\caption{Median Expense Per Persons with Dental Services Expenses}
\label{fig:median_expenses_dental_services}
\end{figure}

\begin{figure}[!ht]
\centering
\includegraphics[width=1\textwidth]{figures/median_expenses/home_health_care_and_other_medical_services_and_equipment}
\caption{Median Expense Per Persons with Home Health Care and Other Medical Services and Equipment Expenses}
\label{fig:median_expenses_home_health_care_and_other_medical_services_and_equipment}
\end{figure}

\begin{figure}[!ht]
\centering
\includegraphics[width=1\textwidth]{figures/insurance_expenses/mean_under_65}
\caption{Mean Expense Per Persons with Under 65 Expenses}
\label{fig:insurance_expenses_mean_under_65}
\end{figure}

\begin{figure}[!ht]
\centering
\includegraphics[width=1\textwidth]{figures/insurance_expenses/median_under_65}
\caption{Median Expense Per Persons with Under 65 Expenses}
\label{fig:insurance_expenses_median_under_65}
\end{figure}

\begin{figure}[!ht]
\centering
\includegraphics[width=1\textwidth]{figures/insurance_expenses/mean_over_65}
\caption{Mean Expense Per Persons with Over 65 Expenses}
\label{fig:insurance_expenses_mean_over_65}
\end{figure}

\begin{figure}[!ht]
\centering
\includegraphics[width=1\textwidth]{figures/insurance_expenses/median_over_65}
\caption{Median Expense Per Persons with Over 65 Expenses}
\label{fig:insurance_expenses_median_over_65}
\end{figure}


%\input{table}

\end{document}

